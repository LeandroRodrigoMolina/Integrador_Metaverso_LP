\documentclass[a4paper,10pt]{article}

% Paquetes requeridos
\usepackage[utf8]{inputenc}
\usepackage[spanish]{babel}
\usepackage{csquotes}
\usepackage{amsmath, amssymb, amsfonts}
\usepackage{graphicx}
\usepackage[style=apa, backend=biber, natbib=true, language=spanish, url=true]{biblatex}
\usepackage{tocloft} % Para personalizar el índice
\usepackage[left=3.5cm,right=2.5cm,top=3.5cm,bottom=3.8cm]{geometry}
\usepackage{setspace} % Espaciado
\usepackage{titlesec} % Para personalizar los títulos
\usepackage{fancyhdr} % Para personalizar encabezados y pies de página
\usepackage{newtxtext}

\pagestyle{fancy}
\fancyhf{} % Limpia encabezados y pies de página
\renewcommand{\headrulewidth}{0pt} % Elimina la línea del encabezado

\addbibresource{referencias.bib}
\DeclareLanguageMapping{spanish}{spanish-apa}
% Configuraciones
\setlength{\parskip}{6pt} % Espacio entre párrafos
\setstretch{1.15} % Espacio entre líneas

\renewcommand{\cftsecleader}{\cftdotfill{\cftdotsep}} % Para puntos en el índice

% Estilos para títulos y subtítulos
\titleformat{\section}
{\normalfont\fontsize{12}{15}\bfseries}{\thesection}{1em}{}
\titleformat{\subsection}
{\normalfont\fontsize{10}{13}\bfseries}{\thesubsection}{1em}{}
\titleformat{\subsubsection}
{\normalfont\fontsize{10.5}{13}\bfseries}{\thesubsubsection}{1em}{}

\usepackage[hypertexnames=false, colorlinks=true, 
linkcolor=blue, 
citecolor=blue, 
urlcolor=blue, 
linkbordercolor={1 1 0}, 
citebordercolor={1 1 0}, 
urlbordercolor={1 1 0}, 
filecolor=blue, 
pdfborderstyle={/S/U/W 1}]{hyperref}

% Inicio del documento
\begin{document}
	\pagestyle{empty}
	% Carátula
	\begin{titlepage}
		\centering
		\vspace*{1.5cm}
		\includegraphics[width=0.3\textwidth]{unerlogo.png}
		\linebreak
		{\fontsize{14}{17}\bfseries Trabajo integrador: Metaverso y Solidity\par}
		{\small Martín Borgo\par}
		{\small Leandro Molina\par}
		{\normalsize Universidad Nacional de Entre Ríos\par}
		{\normalsize Facultad de Ciencias de la Administración\par}
		{\normalsize Licenciatura en Sistemas \par}
		{\small \href{mailto:martinborgo8@gmail.com}{martinborgo8@gmail.com}\par}
		{\small \href{mailto:LeandroRodrigoMolina@gmail.com}{LeandroRodrigoMolina@gmail.com}\par}
		
		% Resumen y palabras clave
		{\small \textbf{Abstract.} Resumen hasta 200 palabras. \par}
		{\small \textbf{Keywords:} Metaverso, Solidity, Realidad Virtual (VR), Realidad Aumentada (AR), Contratos inteligentes (Smart Contracts).\par}
	\end{titlepage}
	
	% Índice
	\tableofcontents
	\thispagestyle{empty}
	\newpage
	%Empezamos a escribir
	\section{Introducción}
	 
	\subsection{Definición del Metaverso}
	El metaverso, conocido también como universo metafísico o espacio virtual, se refiere a un entorno virtual 3D en línea, donde todos los eventos que ocurren en él se producen en tiempo real y tienen un impacto permanente. La palabra “metaverso” está compuesta por el prefijo “meta”, que viene del griego \( \mu \varepsilon \tau \acute{\alpha} \) y significa “más allá” o “después”, y es usado para indicar una abstracción de otro concepto. En el contexto del metaverso, este prefijo se refiere a la idea de un universo que va más allá del universo físico que conocemos. Por otro lado, el final “-verso” proviene del latín “universus”, que significa “todo en uno” o “entero”. En nuestro contexto, sugiere la idea de un espacio completo y autónomo. Por lo tanto, el metaverso se puede interpretar como “un universo alternativo” o “más allá del universo”, refiriéndose a un espacio virtual en línea autónomo que existe más allá de nuestro universo físico.
	
	El metaverso se considera como la revolución de internet, pero, actualmente, no hay una conclusión definitiva sobre su forma final desde todos los ámbitos de convivencia, como el trabajo y la economía, la educación, el entretenimiento, etc. Aunque no haya una conclusión sobre su forma final, es posible describir sus atributos centrales.
	Los atributos centrales del metaverso son los siguientes\footnote{Si desea una descripción más detallada vea el libro de Shenghui Cheng. \cite{cheng2023metaverse}. Pág 2-6.}:
	\begin{itemize}
		\item \textbf{Espacio Sin Límites:} El metaverso es un espacio infinito que debe poder integrarse sin problemas con el mundo real.
		\item \textbf{Persistencia:} El metaverso no termina ni se reinicia, sino que opera de manera abierta y continua, evolucionando indefinidamente.
		\item \textbf{Inmersión:} El metaverso es un espacio virtual con alta inmersión e interactividad.
		\item \textbf{Descentralización:} El metaverso no pertenece a ninguna organización "centralizada", sino que es operado de manera independiente por diferentes participantes.
		\item \textbf{Sistema Económico:} Existe una moneda digital unificada dentro del metaverso, que funciona con un sistema económico virtual impulsado por criptomonedas.
		\item \textbf{Experiencias Sociales:} El metaverso es un mundo co-creado y compartido por todos los participantes, lo cual creará completamente nuevas relaciones y experiencias sociales.
	\end{itemize}
	\subsection{Origen e historia del Metaverso}
	El término metaverso se hizo popular por la novela de Neal Stephenson “Snow Crash”, en esta novela de género ciencia ficción, más específicamente del subgénero Cyberpunk\footnote{El subgénero Cyberpunk se centra en futuros distópicos donde hay tecnología avanzada y todo lo relacionado a la computación está conectado a una sociedad en decadencia. Ejemplos del subgénero Cyberpunk pueden ser: Neuromancer, Akira, Matrix, Cyberpunk 2077, etc.}, 
	aunque no ofrecía una definición clara, sí proporcionaba una interpretación del metaverso. En la novela de Neal, es un espacio virtual donde las personas representadas por avatares, interactúan, socializan y hacen negocios, es una mezcla de realidad física y realidad virtual. En este mundo de ficción, las corporaciones son los proveedores de servicios esenciales dejando de lado al gobierno, y las personas se encuentran en condiciones de vida precarias. Se usa el metaverso para escapar de la realidad, las personas pueden crear avatares siendo versiones idealizadas de ellos mismos, interactuar con entornos limpios y ordenados, participar en actividades que no pueden realizarse en la vida real. Además de utilizarse como medio de comunicación, educación, comercio, etc. La novela no explica la creación de este metaverso, pero podemos inferir que es una respuesta a las condiciones de vida precarias que viven las personas, un tema clásico del Cyberpunk.
	Como se mencionó, el término metaverso se popularizó con “Snow Crash”, pero este término no fue usado por primera vez por Neal este ya existía desde antes, por ejemplo en 1980 William Gibson Neuromancer presentaba un concepto similar. Es interesante la evolución del término metaverso, ya que este nació en la literatura Cyberpunk para los futuros distópicos de sus novelas a evolucionar con el paso del tiempo a una definición formal y con tecnologías en la vida real que ya estamos viendo actualmente.
	
	\subsection{Importancia y futuro del Metaverso}
	La creación de diversas tecnologías que impulsan y adoptan el metaverso como la realidad virtual (VR), realidad aumentada (AR) entre otras. Nos permite ver nuevos mercados y formas de entretenimiento que antes no existían o eran de nicho, el comercio de bienes virtuales, la posesión y alquiler de tierras digitales, son nuevos mercados no explorados, los juegos en VR, vivir eventos históricos, etc. El futuro del metaverso está lleno de retos, desde como una perspectiva de desarrollo (programación), el diseño de los mundos para que estos sean útiles e interesantes. Además de la inclusión de temas relacionados a la ética y la privacidad de los usuarios (los posibles problemas de seguridad que puede tener, ejemplo el hackeo de los periféricos para acceder al metaverso), un problema técnico a revisar es moverse entre los diferentes metaversos, ya que al cada acceso a un metaverso es diferente dificulta que los usuarios se sientan dentro del mismo y la flexibilidad de los mismos.
	\section{Conceptos Clave de Programación para el Metaverso}
	 
	\subsection{Realidad Virtual (VR)}
	Las primeras VR fueron creadas en 1968 por Ivan Sutherland  y David Evans que se llamaron “Espada de Damocles” se consideran las primeras VR porque se conectaban a una computadora y no a una pantalla, esta es la razón por la cual se la piensa como el primer equipo moderno de la realidad virtual. Desde este momento se fue desarrollando esta tecnología (VPL Research se dedicó al estudio de los VR y desarrollaron los primeros sistemas VR comerciales), ya en los 90 surgieron varias empresas que se dedicaban a la fabricación y venta de estos dispositivos. Pero el salto más importante sucedió en 2010 cuando Palmer Luckey fundó la compañía Oculus y revolucionó el mercado con el que sería el mejor casco de VR creado hasta su fecha el Oculus Rift S. Con las características principales de los VR actuales, pantallas de alta resolución, sensores de seguimiento de movimiento precisos y controles de movimiento intuitivos. Esta tecnología llamo la atención de Mark Zuckerberg que, el 25 marzo de 2014, compró a Oculus por 2.000 millones de dólares\footnote{\href{https://about.fb.com/news/2014/03/facebook-to-acquire-oculus/}{Facebook to Acquire Oculus}}
	poco después de haber comprado Whatsapp en 2014 también\footnote{\href{https://about.fb.com/news/2014/02/facebook-to-acquire-whatsapp/}{Facebook to Acquire WhatsApp}}. 
	Con el paso del tiempo desarrollaron las Oculus Quest, Quest2 y, cuando Facebook se pasó a llamar Meta, las Quest Pro. La realidad virtual, se basa en la inmersión del usuario en un entorno digital 3D, para lograr esta sensación, el VR utiliza pantallas estereoscópicas\footnote{De \href{https://dle.rae.es/estereoscopio}{estereoscopio}: aparato en el que, mirando con ambos ojos, se ven dos imágenes de un objeto, que, al fundirse en una, producen una sensación de relieve por estar tomadas con un ángulo diferente para cada ojo.} 
	para proyectar imágenes ligeramente diferentes para cada ojo, estas intentan imitar la percepción de profundidad de los ojos. De manera general tantos los cascos como otros accesorios cuentan con sensores de seguimiento que detectan movimientos de la cabeza, en sistemas más complejos\footnote{Por ejemplo los guantes y trajes hápticos, que permiten al usuario sentir y manipular los objetos virtuales, recreando el tanto, cuando presionas un objeto, vibraciones o movimientos del usuario.}, 
	detectan movimientos del cuerpo entero, para obtener una perspectiva visual en tiempo real más completa por lo tanto más inversiva.
	\subsection{Realidad Aumentada (AR)}
	La realidad aumentada (AR, por sus siglas en inglés), es la capacidad de ciertos dispositivos de insertar imágenes virtuales dentro de un entorno real. En términos más técnicos podemos decir que, combina objetos virtuales con el mundo real superpone imágenes en el mundo real y usando hologramas. La AR y VR son similares en esencia, ya que interactúan con la realidad y lo virtual. Su diferencia con la VR, es que las AR son una ampliación del mundo real, no intenta simular o crear algo completamente diferente. Los dispositivos de AR se dividen en dos categorías: teléfonos AR y gafas inteligentes AR (Estos últimos todavía son emergentes en el mercado). La primera empresa en desarrollar gafas RA fue Google los Google Glass, la idea era (y es) cada lente tiene un proyector incorporado que combina la realidad con lo virtual, su adopción inicial no fue la esperada quedó como tecnología de nicho, con el paso del tiempo se volvió más popular. Hay varias aplicaciones conocidas por los usuarios que son RA ejemplos de estas son: Pokemon Go y los bit-mojis 3D de Snapchat, actualmente esta tecnología tiene diversas aplicaciones en medicina, educación e industria se está demostrando que la AR tiene un impacto más profundo en los humanos como herramienta.
	
	\subsection{Inteligencia Artificial (AI)}
	Máquina
	\subsection{Blockchain y criptomonedas}
	Cadena
	
	\section{Introducción a Solidity y su Relevancia en el Metaverso}
	Plataforma
	\subsection{¿Qué es Solidity?}
	Lenguaje
	\subsection{Conceptos básicos de Solidity: Contratos inteligentes y DApps}
	Automatización
	\subsection{Cómo Solidity impulsa el Metaverso: Casos de uso y aplicaciones}
	Impulso
	\subsection{Desafíos y limitaciones de Solidity en el contexto del Metaverso}
	Desafío
	\subsection{Futuro y evolución de Solidity en el paisaje del Metaverso}
	Evolución
	
	\section{Conclusión}
	Final
	\subsection{Reflexión sobre el estado actual y el futuro del desarrollo del Metaverso}
	Pensamiento
	\subsection{Cómo los lenguajes de programación continuarán evolucionando con el Metaverso}
	Innovación
	
	\nocite{*}
	\printbibliography[heading=bibintoc]
\end{document}
