\documentclass[a4paper,10pt]{article}

% Paquetes requeridos
\usepackage[utf8]{inputenc}
\usepackage[spanish]{babel}
\usepackage{csquotes}
\usepackage{amsmath, amssymb, amsfonts}
\usepackage{graphicx}
\usepackage[style=apa, backend=biber, natbib=true, language=spanish, url=true]{biblatex}
\usepackage{tocloft} % Para personalizar el índice
\usepackage[left=3.5cm,right=2.5cm,top=3.5cm,bottom=3.8cm]{geometry}
\usepackage{setspace} % Espaciado
\usepackage{titlesec} % Para personalizar los títulos
\usepackage{fancyhdr} % Para personalizar encabezados y pies de página
\usepackage{newtxtext}

\pagestyle{fancy}
\fancyhf{} % Limpia encabezados y pies de página
\renewcommand{\headrulewidth}{0pt} % Elimina la línea del encabezado

\addbibresource{referencias.bib}
\DeclareLanguageMapping{spanish}{spanish-apa}
% Configuraciones
\setlength{\parskip}{6pt} % Espacio entre párrafos
\setstretch{1.15} % Espacio entre líneas

\renewcommand{\cftsecleader}{\cftdotfill{\cftdotsep}} % Para puntos en el índice

% Estilos para títulos y subtítulos
\titleformat{\section}
{\normalfont\fontsize{12}{15}\bfseries}{\thesection}{1em}{}
\titleformat{\subsection}
{\normalfont\fontsize{10}{13}\bfseries}{\thesubsection}{1em}{}
\titleformat{\subsubsection}
{\normalfont\fontsize{10.5}{13}\bfseries}{\thesubsubsection}{1em}{}

\usepackage[hypertexnames=false, colorlinks=true, 
linkcolor=blue, 
citecolor=blue, 
urlcolor=blue, 
linkbordercolor={1 1 0}, 
citebordercolor={1 1 0}, 
urlbordercolor={1 1 0}, 
filecolor=blue, 
pdfborderstyle={/S/U/W 1}]{hyperref}

% Inicio del documento
\begin{document}
	\pagestyle{empty}
	% Carátula
	\begin{titlepage}
		\centering
		\vspace*{1.5cm}
		\includegraphics[width=0.3\textwidth]{unerlogo.png}
		\linebreak
		{\fontsize{14}{17}\bfseries Título del Documento\par}
		{\small Martín Borgo\par}
		{\small Leandro Molina\par}
		{\normalsize Universidad Nacional de Entre Ríos\par}
		{\normalsize Facultad de Ciencias de la Administración\par}
		{\normalsize Licenciatura en Sistemas \par}
		{\small \href{mailto:LeandroRodrigoMolina@gmail.com}{Correo electrónico de autor}\par}
		{\small \href{mailto:correo@ejemplo.com}{Correo electrónico de autor}\par}
		
		
		% Resumen y palabras clave con un pequeño desplazamiento hacia la izquierda
		\hspace{-5cm}{\small \textbf{Abstract.} Resumen hasta 200 palabras. \par}
		\hspace{-5.2cm}{\small \textbf{Keywords:} Máximo 5 palabras claves.\par}
	\end{titlepage}
	
	% Índice
	\tableofcontents
	\thispagestyle{empty}
	\newpage
	%Empezamos a escribir
	\section{Introducción}
	Palabra
	\subsection{Definición del Metaverso}
	Aleatorio
	\subsection{Origen e historia del Metaverso}
	Galaxia
	\subsection{Importancia y futuro del Metaverso}
	Futurismo
	
	\section{Conceptos Clave de Programación para el Metaverso}
	Códigos
	\subsection{Realidad Virtual (VR)}
	Virtualidad
	\subsection{Realidad Aumentada (AR)}
	Percepción
	\subsection{Inteligencia Artificial (AI)}
	Máquina
	\subsection{Blockchain y criptomonedas}
	Cadena
	\subsection{IoT (Internet of Things)}
	Conectividad
	
	\section{Introducción a Solidity y su Relevancia en el Metaverso}
	Plataforma
	\subsection{¿Qué es Solidity?}
	Lenguaje
	\subsection{Conceptos básicos de Solidity: Contratos inteligentes y DApps}
	Automatización
	\subsection{Cómo Solidity impulsa el Metaverso: Casos de uso y aplicaciones}
	Impulso
	\subsection{Desafíos y limitaciones de Solidity en el contexto del Metaverso}
	Desafío
	\subsection{Futuro y evolución de Solidity en el paisaje del Metaverso}
	Evolución
	
	\section{Desafíos y Consideraciones en la Programación del Metaverso}
	Programar
	\subsection{Rendimiento y escalabilidad}
	Optimización
	\subsection{Seguridad y privacidad}
	Protección
	\subsection{Interoperabilidad y estándares}
	Estandarización
	\subsection{Consideraciones éticas y sociales}
	Moralidad
	
	\section{Conclusión}
	Final
	\subsection{Reflexión sobre el estado actual y el futuro del desarrollo del Metaverso}
	Pensamiento
	\subsection{Cómo los lenguajes de programación continuarán evolucionando con el Metaverso}
	Innovación

	prueba apa \cite{cheng2023metaverse}
	\printbibliography[heading=bibintoc]
\end{document}
